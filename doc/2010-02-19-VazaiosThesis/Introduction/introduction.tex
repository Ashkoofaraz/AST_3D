\chapter{Introduction}
\label{intro}
Communication is a much need skill, which is met everywhere around us from a human cell and micro-organisms to vastly advanced animals as humans. Even the nature itself reinvents itself if a communication channel is broken down. Humans since the advent of their history form all kinds of groups striving to achieve a common goal, especially , in our time and age ,
%Communication among team members is a crucial ability for any kind of team striving to achieve a common goal, especially
for teams participating in games, where success can only be achieved through collaborative and coordinated efforts.
Teams lacking means of communication are doomed to act simply as a collection of individuals with no added benefit,
other than the multiplicity of individual skills. For human teams, communication is second nature; it is used not only
for teamwork in games, but also in all aspects of life, and takes a multitude of different forms (oral, written,
gestural, visual, auditory). Robots can hardly replicate all human communication means, since these would require
extremely accurate and robust perceptual and action abilities on each robot. Fortunately, most modern robots are capable
of communicating over data networks, an ability which is not available to their human counterparts. However, exploiting
such networking means for team communication purposes in an efficient manner that does not drain the underlying
resources and provides transparent exchange of information in real time is a rather challenging problem. 


This thesis describes a distributed communication framework for robotic teams, which was originally developed for the
Standard Platform League (SPL) of the RoboCup (robotic soccer) competition, but can easily serve any other domain with
similar communication needs. During a RoboCup game, robot players occasionally need to share perceptual, strategic, and
other team-related information with their team-mates in order to coordinate their efforts. In addition, during
development and debugging, human researchers need to be in direct contact with their robots for monitoring and
modification purposes. Typically, these kinds of communication take place over a wireless network. The naive solution of
maintaining a synchronized copy of each robot's data on each node on the network is both inefficient and unrealistic for
such real-time systems. 

Our proposal suggests a distributed and transparent communication framework, called \textit{Narukom}, whereby any node
of the network (robot or remote computer) can access on demand any data available on some other node in a natural and
straightforward way. The framework is based on the publish/subscribe paradigm and provides maximal decoupling not only
between nodes, but also between threads running on the same node. As a result, \textit{Narukom} offers a uniform communication
mechanism between different threads of execution on the same or on different machines. The data shared between the nodes
of the team are stored on local blackboards which are transparently accessible from all nodes. In real-time systems,
such as robotic teams, data come in streams and are being refreshed regularly, therefore it is important to communicate
the latest data or data with a particular time stamp among the robots. To address such synchronization needs, we have
integrated temporal information into the meta-data of the underlying messages exchanged over the network. \textit{Narukom}'s
distributed nature and platform independence make it an ideal base for the development of complex team strategies that
require tight coordination, but also for the distribution of resource-intensive computations, such as learning
experiments, over several (robot and non-robot) nodes.

\section{Thesis Outline}
Chapter~\ref{background} provides some background information on the RoboCup Competition and the underlying technologies used to developed \textit{Narukom}. In Chapter~\ref{communication} we demonstrate some important aspects of the problem of communication across robots. Continuing to chapter~\ref{Narukom}, where the core ideas and an outline of the architecture of our proposal is discussed. Moving on to chapter~\ref{implementation}, a thorough discussion about optimization, implementation design and decisions taken during  the development of \textit{Narukom}. In Chapter \ref{results} a discussion on the results is taking place by providing several experiments in order to evaluate our work. The following chapter ~\ref{related}, presents similar systems developed by other robocup teams including a brief comparison between those systems and ours. Future work and proposals on extending and improving our framework are  the subject of the chapter ~\ref{future}. The final chapter~\ref{conclusion} serves as an epilogue to this thesis, including  a small overview of the system and some long terms plans about \textit{Narukom}.