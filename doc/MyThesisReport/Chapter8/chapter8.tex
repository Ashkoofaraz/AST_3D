\chapter{Conclusion}
\label{conclusion}

Narukom is an attempt to address a lot of communication problems our team had during the last years in Robocup (SPL), while at the same time is a first step towards a cross-platform architecture used in robotic agents. Our journey does not end here our Narukom has, hopefully, a long way, yet to go. The first real world test  is the Robocup 2010 in Singapore where both our inter-process and team communication would be based on our framework.

\section{Out To The Real World}
Narukom is based on open source projects, so the least we could do is to open source it in the near future. We believe that Narukom can contribute to open source community as a cross-platform, distributed, transparent communication framework for everyone willing to use it. I, personally, would be more than happy to assist anyone dare to use our framework. Additionally, any suggestion , comment, critiscism, imporvement and modification is at least welcomed. By open sourcing Narukom we try to achieve two goals take Narukom to the first step and encourage new coders to make available to the community of open source their code in order to advance the collaboration between programmers and avoid reinventing the wheel every now and then. Finally, I would like to thank those who spent some time using Narukom. 


