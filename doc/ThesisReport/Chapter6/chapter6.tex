\chapter{Related Work}
\label{related}

\section{UT Austin Villa}\cite{UTAustinVillaRobotSoccerTeam}
UT Austin Villa is the most known and the best team which is participating in the Robocup's simulation league. Its first appearance was in the Robocup 2007 held in Atlanta, U.S.A, in July 2007. It belongs to University of Texas and consists of five members, professor Peter Stone, 
graduate students Patrick MacAlpine and Samuel Barrett and finally two undergraduate students Nick Collins and Adrian Lopez-Mobilia. The main characteristic of this team is its state-of-art dynamic movement. Its fast and stable walk is recognizable and offers them great results. A typical example of this great team's results can be that in Robocup's competition in Istanbul 2011 this team won all 24 games it played and scored a total of 136 goals without conceding any.

In their last paper \cite{UtAustinVillaPaper} about positioning, it is explained their approach of player positioning in the field. First, a full team formation is computed. Second, each players calculates the best assignment of players according to his belief about the world. Finally, a coordination mechanism is used to choose among all players' suggestions. This coordination mechanism using a voting system. Players assignment with the most votes will be used as a result.

I am not the appropriate person to criticize their longterm work and contribution to the Robocup's simulation league. However, I would like to mention that in our approach there is a major difference in the way that players coordinate their actions. The separation of the team into subsets makes it easier to solve all these problem caused due to complexity constraints. Furthermore, we are using active's group players in order to have a better role assignment to positions near to ball which have huge importance in games like soccer. Finally, I wish we had such a perfect movement controller like UT Austin Villa's one. It would be a nice challenge to compare these two coordination systems in the same movements' level.


\section{BeeStanbul}
\cite{BeeStanbulTDP} The beeStanbul project from the Artificial Intelligence and Robotics laboratory
(AIR lab) at Istanbul Technical University (ITU) is the first initiative from ITU
to participate in RoboCup competitions. It consists of five members and has been participating in the Robocup's competitions since RoboCup 2010 held in Singapore. It is a nice team which accomplished to qualify up to second round in the last Robocup competition in Mexico 2012.

First of all, they are making use of both static and dynamic movements and their walking machine is more than adequate. Concentrating in their work at the coordination part of their project.They split team agents into three groups, defenders and attackers. The attackers group involves the forward and the midfielder agents while the defend-
ers group involves only the defender agents.Since two agents are assigned to the
goalkeeper and the forward roles, the remaining seven agents are to be assigned
to these roles. This
is accomplished by a distributed Voronoi cell construction approach in which
each agent calculates its own cell independently from that of the others. Therefore, every agent has a differently shaped cell and these can overlap. The time complexity of the method is $O(n^{2})$ where n is the number of agents in the team. After constructing the cell
for itself, each agent determines the center of the cell as its new target. Agents
become closer to each other by using this strategy. In their approach, only teammates in the viewpoint of the agent are considered. So we can realize that there can be situation in a soccer game when each agent who computes his own cell could be completely unaware if any of his teammates is located in his field of view. Having a better knowledge of teammates position in the field was a key feature in our approach.


\section{RoboCanes}







