\chapter{Introduction}
\label{intro}
What will happen if we place a team of robots into a soccer field? It is obvious for everyone to realise that nothing is going to happen. This occurs due to the fact that, machines such as robots should be programmed to percept their surroundings and act just like a human soccer players. Therefore, everything in the robots' world, start from the absolute zero. Even if, these robots had a perfect sense of their environment, it would be difficult for them to start playing soccer immediately. There are plenty of things that have to be done until these robots start playing in the way human players do. Starting from the beginning there must be a connection with the simulation server; Server sends to each connected agent messages every 20ms, these messages include information about agent's sight and other perceptions. Each agent use these messages to update his perceptions. At the end of the scanning of each message the agent knows the values of every joint of his body. He has also knowledge about the location in relation to his body of every landmark, the ball and other players which are in his sight's field. Having this knowledge we can proceed to more complicated things. First of all, agents have to calculate their position in the soccer field, it is not so simple as it sounds and it requires at least two landmarks in our sight's field. We are going to explain this operation extensively later. Even if, these agents know their positions in the soccer field and can calculate the position of every other agent in their sight  as well as the soccer ball position, they are still not able to perform a single action. This will be feasible if they combinate motions which are going to help them perform each action. Even in real life, a soccer player  has to combine simple movements for example, walking, turning and kicking, to perform a kick towards the opponents' goal. The same principle applies in simulation soccer too. In our approach, we have categorize the actions in relation to their complexity. At first simple actions, which just use motions in order to be completed. On the other hand complex actions make use of more than one simple actions to be performed with success. An example of a simple action is a turn towards the ball and a more complex action could be walking to a specific coordinate in the soccer field. We can realize that a complex action such as the above is going to make use of other more simple actions. Until now, we have accomplished every agent in the field to be able to recognize objects, find its position in the soccer field and do simple and complex actions.Returning to the first question which we have put in the  beginning of this introduction, we could answer with certainty that every agent in the soccer field has a complete sense of its environment and is able to perform actions. This is not going to bring success to the team, agents have not the ability to communicate with their team-mates and reasonably they are not able to coordinate their actions. Even humans since the advent of their history form all kinds of groups striving to achieve a common goal, especially , in our time and age ,for teams participating in games, where success can only be achieved through collaborative and coordinated efforts. This is going to be accomplished through communication. This thesis describes a way of making the coordination of the agents through communication as well as a proposed solution of all the problems generated in robotic soccer. 
The main objective of the this thesis is to develop an efficient software system to correctly model the behaviors of simulated
Nao robots in such a competitive environment as simulation soccer league. The challenging and the most time consuming part of this project was the coordination part which I firmly believe that is the most important part either in a simulated team or in a real soccer team.



\section{Thesis Outline}